
% Default to the notebook output style

    


% Inherit from the specified cell style.




    
\documentclass{article}

    
    
    \usepackage{graphicx} % Used to insert images
    \usepackage{adjustbox} % Used to constrain images to a maximum size 
    \usepackage{color} % Allow colors to be defined
    \usepackage{enumerate} % Needed for markdown enumerations to work
    \usepackage{geometry} % Used to adjust the document margins
    \usepackage{amsmath} % Equations
    \usepackage{amssymb} % Equations
    \usepackage{eurosym} % defines \euro
    \usepackage[mathletters]{ucs} % Extended unicode (utf-8) support
    \usepackage[utf8x]{inputenc} % Allow utf-8 characters in the tex document
    \usepackage{fancyvrb} % verbatim replacement that allows latex
    \usepackage{grffile} % extends the file name processing of package graphics 
                         % to support a larger range 
    % The hyperref package gives us a pdf with properly built
    % internal navigation ('pdf bookmarks' for the table of contents,
    % internal cross-reference links, web links for URLs, etc.)
    \usepackage{hyperref}
    \usepackage{longtable} % longtable support required by pandoc >1.10
    \usepackage{booktabs}  % table support for pandoc > 1.12.2
    

    
    
    \definecolor{orange}{cmyk}{0,0.4,0.8,0.2}
    \definecolor{darkorange}{rgb}{.71,0.21,0.01}
    \definecolor{darkgreen}{rgb}{.12,.54,.11}
    \definecolor{myteal}{rgb}{.26, .44, .56}
    \definecolor{gray}{gray}{0.45}
    \definecolor{lightgray}{gray}{.95}
    \definecolor{mediumgray}{gray}{.8}
    \definecolor{inputbackground}{rgb}{.95, .95, .85}
    \definecolor{outputbackground}{rgb}{.95, .95, .95}
    \definecolor{traceback}{rgb}{1, .95, .95}
    % ansi colors
    \definecolor{red}{rgb}{.6,0,0}
    \definecolor{green}{rgb}{0,.65,0}
    \definecolor{brown}{rgb}{0.6,0.6,0}
    \definecolor{blue}{rgb}{0,.145,.698}
    \definecolor{purple}{rgb}{.698,.145,.698}
    \definecolor{cyan}{rgb}{0,.698,.698}
    \definecolor{lightgray}{gray}{0.5}
    
    % bright ansi colors
    \definecolor{darkgray}{gray}{0.25}
    \definecolor{lightred}{rgb}{1.0,0.39,0.28}
    \definecolor{lightgreen}{rgb}{0.48,0.99,0.0}
    \definecolor{lightblue}{rgb}{0.53,0.81,0.92}
    \definecolor{lightpurple}{rgb}{0.87,0.63,0.87}
    \definecolor{lightcyan}{rgb}{0.5,1.0,0.83}
    
    % commands and environments needed by pandoc snippets
    % extracted from the output of `pandoc -s`
    \DefineVerbatimEnvironment{Highlighting}{Verbatim}{commandchars=\\\{\}}
    % Add ',fontsize=\small' for more characters per line
    \newenvironment{Shaded}{}{}
    \newcommand{\KeywordTok}[1]{\textcolor[rgb]{0.00,0.44,0.13}{\textbf{{#1}}}}
    \newcommand{\DataTypeTok}[1]{\textcolor[rgb]{0.56,0.13,0.00}{{#1}}}
    \newcommand{\DecValTok}[1]{\textcolor[rgb]{0.25,0.63,0.44}{{#1}}}
    \newcommand{\BaseNTok}[1]{\textcolor[rgb]{0.25,0.63,0.44}{{#1}}}
    \newcommand{\FloatTok}[1]{\textcolor[rgb]{0.25,0.63,0.44}{{#1}}}
    \newcommand{\CharTok}[1]{\textcolor[rgb]{0.25,0.44,0.63}{{#1}}}
    \newcommand{\StringTok}[1]{\textcolor[rgb]{0.25,0.44,0.63}{{#1}}}
    \newcommand{\CommentTok}[1]{\textcolor[rgb]{0.38,0.63,0.69}{\textit{{#1}}}}
    \newcommand{\OtherTok}[1]{\textcolor[rgb]{0.00,0.44,0.13}{{#1}}}
    \newcommand{\AlertTok}[1]{\textcolor[rgb]{1.00,0.00,0.00}{\textbf{{#1}}}}
    \newcommand{\FunctionTok}[1]{\textcolor[rgb]{0.02,0.16,0.49}{{#1}}}
    \newcommand{\RegionMarkerTok}[1]{{#1}}
    \newcommand{\ErrorTok}[1]{\textcolor[rgb]{1.00,0.00,0.00}{\textbf{{#1}}}}
    \newcommand{\NormalTok}[1]{{#1}}
    
    % Define a nice break command that doesn't care if a line doesn't already
    % exist.
    \def\br{\hspace*{\fill} \\* }
    % Math Jax compatability definitions
    \def\gt{>}
    \def\lt{<}
    % Document parameters
    \title{Fonctions de probabilité de densité et coefficients de corrélation comme outils d'analyse mélodique}
    \author{Anas GHRAB - http://anas.ghrab.tn\\
    Institut Supérieur de Musique - Université de Sousse (Tunisie)\\
    Centre des Musiques Arabes et Méditerranéennes}
    \date{31 mai 2015}    
    

    % Pygments definitions
    
\makeatletter
\def\PY@reset{\let\PY@it=\relax \let\PY@bf=\relax%
    \let\PY@ul=\relax \let\PY@tc=\relax%
    \let\PY@bc=\relax \let\PY@ff=\relax}
\def\PY@tok#1{\csname PY@tok@#1\endcsname}
\def\PY@toks#1+{\ifx\relax#1\empty\else%
    \PY@tok{#1}\expandafter\PY@toks\fi}
\def\PY@do#1{\PY@bc{\PY@tc{\PY@ul{%
    \PY@it{\PY@bf{\PY@ff{#1}}}}}}}
\def\PY#1#2{\PY@reset\PY@toks#1+\relax+\PY@do{#2}}

\expandafter\def\csname PY@tok@gd\endcsname{\def\PY@tc##1{\textcolor[rgb]{0.63,0.00,0.00}{##1}}}
\expandafter\def\csname PY@tok@gu\endcsname{\let\PY@bf=\textbf\def\PY@tc##1{\textcolor[rgb]{0.50,0.00,0.50}{##1}}}
\expandafter\def\csname PY@tok@gt\endcsname{\def\PY@tc##1{\textcolor[rgb]{0.00,0.27,0.87}{##1}}}
\expandafter\def\csname PY@tok@gs\endcsname{\let\PY@bf=\textbf}
\expandafter\def\csname PY@tok@gr\endcsname{\def\PY@tc##1{\textcolor[rgb]{1.00,0.00,0.00}{##1}}}
\expandafter\def\csname PY@tok@cm\endcsname{\let\PY@it=\textit\def\PY@tc##1{\textcolor[rgb]{0.25,0.50,0.50}{##1}}}
\expandafter\def\csname PY@tok@vg\endcsname{\def\PY@tc##1{\textcolor[rgb]{0.10,0.09,0.49}{##1}}}
\expandafter\def\csname PY@tok@m\endcsname{\def\PY@tc##1{\textcolor[rgb]{0.40,0.40,0.40}{##1}}}
\expandafter\def\csname PY@tok@mh\endcsname{\def\PY@tc##1{\textcolor[rgb]{0.40,0.40,0.40}{##1}}}
\expandafter\def\csname PY@tok@go\endcsname{\def\PY@tc##1{\textcolor[rgb]{0.53,0.53,0.53}{##1}}}
\expandafter\def\csname PY@tok@ge\endcsname{\let\PY@it=\textit}
\expandafter\def\csname PY@tok@vc\endcsname{\def\PY@tc##1{\textcolor[rgb]{0.10,0.09,0.49}{##1}}}
\expandafter\def\csname PY@tok@il\endcsname{\def\PY@tc##1{\textcolor[rgb]{0.40,0.40,0.40}{##1}}}
\expandafter\def\csname PY@tok@cs\endcsname{\let\PY@it=\textit\def\PY@tc##1{\textcolor[rgb]{0.25,0.50,0.50}{##1}}}
\expandafter\def\csname PY@tok@cp\endcsname{\def\PY@tc##1{\textcolor[rgb]{0.74,0.48,0.00}{##1}}}
\expandafter\def\csname PY@tok@gi\endcsname{\def\PY@tc##1{\textcolor[rgb]{0.00,0.63,0.00}{##1}}}
\expandafter\def\csname PY@tok@gh\endcsname{\let\PY@bf=\textbf\def\PY@tc##1{\textcolor[rgb]{0.00,0.00,0.50}{##1}}}
\expandafter\def\csname PY@tok@ni\endcsname{\let\PY@bf=\textbf\def\PY@tc##1{\textcolor[rgb]{0.60,0.60,0.60}{##1}}}
\expandafter\def\csname PY@tok@nl\endcsname{\def\PY@tc##1{\textcolor[rgb]{0.63,0.63,0.00}{##1}}}
\expandafter\def\csname PY@tok@nn\endcsname{\let\PY@bf=\textbf\def\PY@tc##1{\textcolor[rgb]{0.00,0.00,1.00}{##1}}}
\expandafter\def\csname PY@tok@no\endcsname{\def\PY@tc##1{\textcolor[rgb]{0.53,0.00,0.00}{##1}}}
\expandafter\def\csname PY@tok@na\endcsname{\def\PY@tc##1{\textcolor[rgb]{0.49,0.56,0.16}{##1}}}
\expandafter\def\csname PY@tok@nb\endcsname{\def\PY@tc##1{\textcolor[rgb]{0.00,0.50,0.00}{##1}}}
\expandafter\def\csname PY@tok@nc\endcsname{\let\PY@bf=\textbf\def\PY@tc##1{\textcolor[rgb]{0.00,0.00,1.00}{##1}}}
\expandafter\def\csname PY@tok@nd\endcsname{\def\PY@tc##1{\textcolor[rgb]{0.67,0.13,1.00}{##1}}}
\expandafter\def\csname PY@tok@ne\endcsname{\let\PY@bf=\textbf\def\PY@tc##1{\textcolor[rgb]{0.82,0.25,0.23}{##1}}}
\expandafter\def\csname PY@tok@nf\endcsname{\def\PY@tc##1{\textcolor[rgb]{0.00,0.00,1.00}{##1}}}
\expandafter\def\csname PY@tok@si\endcsname{\let\PY@bf=\textbf\def\PY@tc##1{\textcolor[rgb]{0.73,0.40,0.53}{##1}}}
\expandafter\def\csname PY@tok@s2\endcsname{\def\PY@tc##1{\textcolor[rgb]{0.73,0.13,0.13}{##1}}}
\expandafter\def\csname PY@tok@vi\endcsname{\def\PY@tc##1{\textcolor[rgb]{0.10,0.09,0.49}{##1}}}
\expandafter\def\csname PY@tok@nt\endcsname{\let\PY@bf=\textbf\def\PY@tc##1{\textcolor[rgb]{0.00,0.50,0.00}{##1}}}
\expandafter\def\csname PY@tok@nv\endcsname{\def\PY@tc##1{\textcolor[rgb]{0.10,0.09,0.49}{##1}}}
\expandafter\def\csname PY@tok@s1\endcsname{\def\PY@tc##1{\textcolor[rgb]{0.73,0.13,0.13}{##1}}}
\expandafter\def\csname PY@tok@kd\endcsname{\let\PY@bf=\textbf\def\PY@tc##1{\textcolor[rgb]{0.00,0.50,0.00}{##1}}}
\expandafter\def\csname PY@tok@sh\endcsname{\def\PY@tc##1{\textcolor[rgb]{0.73,0.13,0.13}{##1}}}
\expandafter\def\csname PY@tok@sc\endcsname{\def\PY@tc##1{\textcolor[rgb]{0.73,0.13,0.13}{##1}}}
\expandafter\def\csname PY@tok@sx\endcsname{\def\PY@tc##1{\textcolor[rgb]{0.00,0.50,0.00}{##1}}}
\expandafter\def\csname PY@tok@bp\endcsname{\def\PY@tc##1{\textcolor[rgb]{0.00,0.50,0.00}{##1}}}
\expandafter\def\csname PY@tok@c1\endcsname{\let\PY@it=\textit\def\PY@tc##1{\textcolor[rgb]{0.25,0.50,0.50}{##1}}}
\expandafter\def\csname PY@tok@kc\endcsname{\let\PY@bf=\textbf\def\PY@tc##1{\textcolor[rgb]{0.00,0.50,0.00}{##1}}}
\expandafter\def\csname PY@tok@c\endcsname{\let\PY@it=\textit\def\PY@tc##1{\textcolor[rgb]{0.25,0.50,0.50}{##1}}}
\expandafter\def\csname PY@tok@mf\endcsname{\def\PY@tc##1{\textcolor[rgb]{0.40,0.40,0.40}{##1}}}
\expandafter\def\csname PY@tok@err\endcsname{\def\PY@bc##1{\setlength{\fboxsep}{0pt}\fcolorbox[rgb]{1.00,0.00,0.00}{1,1,1}{\strut ##1}}}
\expandafter\def\csname PY@tok@mb\endcsname{\def\PY@tc##1{\textcolor[rgb]{0.40,0.40,0.40}{##1}}}
\expandafter\def\csname PY@tok@ss\endcsname{\def\PY@tc##1{\textcolor[rgb]{0.10,0.09,0.49}{##1}}}
\expandafter\def\csname PY@tok@sr\endcsname{\def\PY@tc##1{\textcolor[rgb]{0.73,0.40,0.53}{##1}}}
\expandafter\def\csname PY@tok@mo\endcsname{\def\PY@tc##1{\textcolor[rgb]{0.40,0.40,0.40}{##1}}}
\expandafter\def\csname PY@tok@kn\endcsname{\let\PY@bf=\textbf\def\PY@tc##1{\textcolor[rgb]{0.00,0.50,0.00}{##1}}}
\expandafter\def\csname PY@tok@mi\endcsname{\def\PY@tc##1{\textcolor[rgb]{0.40,0.40,0.40}{##1}}}
\expandafter\def\csname PY@tok@gp\endcsname{\let\PY@bf=\textbf\def\PY@tc##1{\textcolor[rgb]{0.00,0.00,0.50}{##1}}}
\expandafter\def\csname PY@tok@o\endcsname{\def\PY@tc##1{\textcolor[rgb]{0.40,0.40,0.40}{##1}}}
\expandafter\def\csname PY@tok@kr\endcsname{\let\PY@bf=\textbf\def\PY@tc##1{\textcolor[rgb]{0.00,0.50,0.00}{##1}}}
\expandafter\def\csname PY@tok@s\endcsname{\def\PY@tc##1{\textcolor[rgb]{0.73,0.13,0.13}{##1}}}
\expandafter\def\csname PY@tok@kp\endcsname{\def\PY@tc##1{\textcolor[rgb]{0.00,0.50,0.00}{##1}}}
\expandafter\def\csname PY@tok@w\endcsname{\def\PY@tc##1{\textcolor[rgb]{0.73,0.73,0.73}{##1}}}
\expandafter\def\csname PY@tok@kt\endcsname{\def\PY@tc##1{\textcolor[rgb]{0.69,0.00,0.25}{##1}}}
\expandafter\def\csname PY@tok@ow\endcsname{\let\PY@bf=\textbf\def\PY@tc##1{\textcolor[rgb]{0.67,0.13,1.00}{##1}}}
\expandafter\def\csname PY@tok@sb\endcsname{\def\PY@tc##1{\textcolor[rgb]{0.73,0.13,0.13}{##1}}}
\expandafter\def\csname PY@tok@k\endcsname{\let\PY@bf=\textbf\def\PY@tc##1{\textcolor[rgb]{0.00,0.50,0.00}{##1}}}
\expandafter\def\csname PY@tok@se\endcsname{\let\PY@bf=\textbf\def\PY@tc##1{\textcolor[rgb]{0.73,0.40,0.13}{##1}}}
\expandafter\def\csname PY@tok@sd\endcsname{\let\PY@it=\textit\def\PY@tc##1{\textcolor[rgb]{0.73,0.13,0.13}{##1}}}

\def\PYZbs{\char`\\}
\def\PYZus{\char`\_}
\def\PYZob{\char`\{}
\def\PYZcb{\char`\}}
\def\PYZca{\char`\^}
\def\PYZam{\char`\&}
\def\PYZlt{\char`\<}
\def\PYZgt{\char`\>}
\def\PYZsh{\char`\#}
\def\PYZpc{\char`\%}
\def\PYZdl{\char`\$}
\def\PYZhy{\char`\-}
\def\PYZsq{\char`\'}
\def\PYZdq{\char`\"}
\def\PYZti{\char`\~}
% for compatibility with earlier versions
\def\PYZat{@}
\def\PYZlb{[}
\def\PYZrb{]}
\makeatother


    % Exact colors from NB
    \definecolor{incolor}{rgb}{0.0, 0.0, 0.5}
    \definecolor{outcolor}{rgb}{0.545, 0.0, 0.0}



    
    % Prevent overflowing lines due to hard-to-break entities
    \sloppy 
    % Setup hyperref package
    \hypersetup{
      breaklinks=true,  % so long urls are correctly broken across lines
      colorlinks=true,
      urlcolor=blue,
      linkcolor=darkorange,
      citecolor=darkgreen,
      }
    % Slightly bigger margins than the latex defaults
    
    \geometry{verbose,tmargin=1in,bmargin=1in,lmargin=1in,rmargin=1in}
    
    

    \begin{document}
    
    
    \maketitle
    

    \subsection*{Introduction}\label{introduction}

    Un des objectifs de l'analyse de la musique modale est la comparaison
des mélodies, ce qui implique, progressivement, leur classification.
Alors que les méthodes classiques d'analyse permettent de saisir les
spécificités modales sur un ensemble délimité, et dont le nombre est
généralement réduit, nous savons que la pertinence des résultats est en
étroite relation avec la quantité de données analysées.

Nous proposons ici une méthode générale pour la mesure de la proximité
mélodique applicable à une base de données sonores. Nous partons d'une
analyse des données fréquentielles afin d'obtenir une fonction indiquant
la répartition de la densité des fréquences. Ceci nous permettra
d'aboutir, par le calcul des différents coefficients de corrélation de
ces fonctions, à une classification des mélodies, nous renseignant ainsi
sur leur proximité mélodique, sur la base de leur échelle.

Nous montrons ici l'aspect pratique de cette approche, en prenant comme
exemple l'analyse d'une chant à voix seule par l'illustre \emph{Muqriʾ}
tunisien ʿAlī al-Barrāq (1891-1981).

L'ensemble des approches présentées ici ont été implémenté dans un
module Python, que nous avons appelé \textbf{Diastema}. Ce module est
téléchargeable à partir de l'adresse :
https://github.com/AnasGhrab/diastema. Nous incluons ici toutes les
commandes nécessaires à son utilisation.

    \begin{Verbatim}[commandchars=\\\{\}]
{\color{incolor}In [{\color{incolor}1}]:} \PY{o}{\PYZpc{}}\PY{k}{matplotlib} inline
        \PY{k+kn}{from} \PY{n+nn}{diastema} \PY{k+kn}{import} \PY{o}{*}
        \PY{k+kn}{from} \PY{n+nn}{IPython.display} \PY{k+kn}{import} \PY{n}{Audio}
\end{Verbatim}

    \section{L'échelle musicale représentée par une fonction de densité de
probabilité des
fréquences}\label{luxe9chelle-musicale-repruxe9sentuxe9e-par-une-fonction-de-densituxe9-de-probabilituxe9-des-fruxe9quences}

    \subsection{La mélodie comme liste de
fréquences}\label{la-muxe9lodie-comme-liste-de-fruxe9quences}

    On sait, au moins depuis les travaux de Charles Seeger (1886 -- 1979)
sur le mélographe, qu'il est musicologiquement plus pertinant de
représenter une mélodie par une courbe fréquencielle qu'en utilisant une
notation symbolique (voir fig. 1).

    \begin{Verbatim}[commandchars=\\\{\}]
{\color{incolor}In [{\color{incolor}2}]:} \PY{k+kn}{from} \PY{n+nn}{IPython.display} \PY{k+kn}{import} \PY{n}{Image}
        
        \PY{n}{Image}\PY{p}{(}\PY{n}{url}\PY{o}{=}\PY{l+s}{\PYZsq{}}\PY{l+s}{http://localhost/images/melograph.jpg}\PY{l+s}{\PYZsq{}}\PY{p}{)}
\end{Verbatim}

            \begin{Verbatim}[commandchars=\\\{\}]
{\color{outcolor}Out[{\color{outcolor}2}]:} <IPython.core.display.Image object>
\end{Verbatim}
        
    \begin{Verbatim}[commandchars=\\\{\}]
{\color{incolor}In [{\color{incolor}3}]:} \PY{n}{Image}\PY{p}{(}\PY{n}{url}\PY{o}{=}\PY{l+s}{\PYZsq{}}\PY{l+s}{http://localhost/images/seegermelo.jpg}\PY{l+s}{\PYZsq{}}\PY{p}{)}
\end{Verbatim}

            \begin{Verbatim}[commandchars=\\\{\}]
{\color{outcolor}Out[{\color{outcolor}3}]:} <IPython.core.display.Image object>
\end{Verbatim}
        
    Les données fréquencielles d'une mélodie peuvent être obtenues
aujourd'hui par l'analyse computationnelle de la fondamentale d'un
enregistrement sonore numérique. Plusieurs algorithmes et
implémentations existent : l'algorithme de Praat (Paul Boersma :
\emph{Accurate short-term analysis of the fundamental frequency and the
harmonics-to-noise ratio of a sampled sound.} IFA Proceedings 17(1993):
97-110.), Yin (ref), etc. Nous utiliserons ici l'algorithme
\textbf{Melodia} implémenté dans Essentia par la méthode
PredominentMelody() (ref).

    Prenons comme exemple un chant en \emph{Raṣd Ḏīl} de ʿAlī Barrāq
(enregistrement publié dans le CD accompagnant l'ouvrage : Manoubi
Snoussi, \emph{Initiation à la Musique Tunisienne}, CMAM, \ldots{}),
segmenté manuellement en six phrases :

    \begin{Verbatim}[commandchars=\\\{\}]
{\color{incolor}In [{\color{incolor}4}]:} \PY{n}{dossier} \PY{o}{=} \PY{l+s}{\PYZdq{}}\PY{l+s}{/Users/anas/AUDIO/Barraq/}\PY{l+s}{\PYZdq{}}
        \PY{n}{Barraq} \PY{o}{=} \PY{n}{Melodies}\PY{p}{(}\PY{n}{dossier}\PY{p}{,}\PY{l+m+mi}{100}\PY{p}{,}\PY{l+m+mi}{450}\PY{p}{)}
\end{Verbatim}

    \begin{Verbatim}[commandchars=\\\{\}]
Lecture et analyse de  7  fichiers (.txt) dans le dossier : /Users/anas/AUDIO/Barraq/
    \end{Verbatim}

    On peut représenter par exemple la courbe mélodique de la première
phrase :

    \begin{Verbatim}[commandchars=\\\{\}]
{\color{incolor}In [{\color{incolor}5}]:} \PY{n}{Y} \PY{o}{=} \PY{n}{Barraq}\PY{o}{.}\PY{n}{melodies}\PY{p}{[}\PY{l+m+mi}{1}\PY{p}{]}\PY{o}{.}\PY{n}{frequences} \PY{c}{\PYZsh{} La numérotation des mélodies commence à zéro}
        \PY{n}{plt}\PY{o}{.}\PY{n}{figure}\PY{p}{(}\PY{n}{figsize}\PY{o}{=}\PY{p}{(}\PY{l+m+mi}{9}\PY{p}{,}\PY{l+m+mi}{6}\PY{p}{)}\PY{p}{)}
        \PY{n}{plt}\PY{o}{.}\PY{n}{plot}\PY{p}{(}\PY{n}{Y}\PY{p}{,}\PY{l+s}{\PYZsq{}}\PY{l+s}{o}\PY{l+s}{\PYZsq{}}\PY{p}{)}
\end{Verbatim}

            \begin{Verbatim}[commandchars=\\\{\}]
{\color{outcolor}Out[{\color{outcolor}5}]:} [<matplotlib.lines.Line2D at 0x10bae21d0>]
\end{Verbatim}
        
    \begin{center}
    \adjustimage{max size={0.9\linewidth}{0.9\paperheight}}{barraq_files/barraq_14_1.png}
    \end{center}
    { \hspace*{\fill} \\}
    
    \begin{Verbatim}[commandchars=\\\{\}]
{\color{incolor}In [{\color{incolor}6}]:} \PY{n}{Audio}\PY{p}{(}\PY{n}{filename}\PY{o}{=}\PY{l+s}{\PYZdq{}}\PY{l+s}{/Users/anas/AUDIO/Barraq/P1.wav}\PY{l+s}{\PYZdq{}}\PY{p}{)}
\end{Verbatim}

            \begin{Verbatim}[commandchars=\\\{\}]
{\color{outcolor}Out[{\color{outcolor}6}]:} <IPython.lib.display.Audio object>
\end{Verbatim}
        
    Analyser la courbe revient à analyser les données numériques de la
courbe.

    \begin{Verbatim}[commandchars=\\\{\}]
{\color{incolor}In [{\color{incolor}7}]:} \PY{n}{Y} \PY{o}{=} \PY{n}{Barraq}\PY{o}{.}\PY{n}{melodies}\PY{p}{[}\PY{l+m+mi}{1}\PY{p}{]}\PY{o}{.}\PY{n}{freq}
        \PY{k}{print} \PY{n}{Y}\PY{p}{[}\PY{l+m+mi}{0}\PY{p}{:}\PY{l+m+mi}{50}\PY{p}{]}
\end{Verbatim}

    \begin{Verbatim}[commandchars=\\\{\}]
[ 155.564  157.372  160.123  163.866  166.73   168.667  171.616  173.61
  175.627  177.668  179.732  180.773  182.874  184.999  186.07   188.232
  189.323  189.323  191.523  192.632  193.748  193.748  194.87   194.87
  194.87   194.87   194.87   194.87   193.748  193.748  192.632  192.632
  193.748  193.748  194.87   194.87   195.999  195.999  195.999  195.999
  194.87   194.87   194.87   194.87   194.87   194.87   194.87   194.87
  194.87   194.87 ]
    \end{Verbatim}

    \subsection{Fonction de densité de probabilité et pôles
fréquentiels}\label{fonction-de-densituxe9-de-probabilituxe9-et-puxf4les-fruxe9quentiels}

    Une des méthodes permettant d'analyser la densité des fréquences dans
les données est l'estimation de la densité par noyaux gaussiens
(\emph{Kernel Density Estimation} :
http://docs.scipy.org/doc/scipy-0.15.1/reference/generated/scipy.stats.gaussian\_kde.html;
nous utiliserons ici un bw\_method=.1). Cette méthode nous permet
d'avoir une fonction de densité de probabilité (\emph{Probability
Density Estimation}). Appliquée à la première phrase, nous obtenons la
représentation suivante :

    \begin{Verbatim}[commandchars=\\\{\}]
{\color{incolor}In [{\color{incolor}8}]:} \PY{n}{plt}\PY{o}{.}\PY{n}{figure}\PY{p}{(}\PY{n}{figsize}\PY{o}{=}\PY{p}{(}\PY{l+m+mi}{9}\PY{p}{,}\PY{l+m+mi}{6}\PY{p}{)}\PY{p}{)}
        \PY{n}{Barraq}\PY{o}{.}\PY{n}{melodies}\PY{p}{[}\PY{l+m+mi}{0}\PY{p}{]}\PY{o}{.}\PY{n}{pdf\PYZus{}show}\PY{p}{(}\PY{p}{)}
\end{Verbatim}

    \begin{center}
    \adjustimage{max size={0.9\linewidth}{0.9\paperheight}}{barraq_files/barraq_20_0.png}
    \end{center}
    { \hspace*{\fill} \\}
    
    Nous pouvons lire sur le graphe que la fréquence dont l'appaarition est
la plus probable dans un pareil extrait, c'est la fréquence 242 Hz.
Ensuite, avec une moindre probabilité, la fréquence 200 Hz, etc. Il est
possible d'obtenir tous les pics représentés par la commande :

    \begin{Verbatim}[commandchars=\\\{\}]
{\color{incolor}In [{\color{incolor}9}]:} \PY{n}{Barraq}\PY{o}{.}\PY{n}{melodies}\PY{p}{[}\PY{l+m+mi}{0}\PY{p}{]}\PY{o}{.}\PY{n}{peaks}
\end{Verbatim}

            \begin{Verbatim}[commandchars=\\\{\}]
{\color{outcolor}Out[{\color{outcolor}9}]:} array([ 162.1777,  200.2865,  219.341 ,  242.4069,  273.4957,  296.5616,
                321.6332,  365.7593,  404.8711,  434.957 ])
\end{Verbatim}
        
    Les probabilités de ces pics sont :

    \begin{Verbatim}[commandchars=\\\{\}]
{\color{incolor}In [{\color{incolor}10}]:} \PY{n}{Barraq}\PY{o}{.}\PY{n}{melodies}\PY{p}{[}\PY{l+m+mi}{0}\PY{p}{]}\PY{o}{.}\PY{n}{peakspdf}
\end{Verbatim}

            \begin{Verbatim}[commandchars=\\\{\}]
{\color{outcolor}Out[{\color{outcolor}10}]:} array([  5.2252e-03,   8.5358e-03,   5.0912e-03,   1.0044e-02,
                  5.6737e-03,   4.4165e-03,   4.9976e-03,   1.1381e-03,
                  2.2812e-04,   2.8166e-05])
\end{Verbatim}
        
    \subsection{Comparaison des fonctions de densité de
probabilité}\label{comparaison-des-fonctions-de-densituxe9-de-probabilituxe9}

    Nous pouvons généraliser la procédure et placer sur un même graphe les
fonctions de densités de probabilités relatives aux fréquences des
quatre phrases musicales :

    \begin{Verbatim}[commandchars=\\\{\}]
{\color{incolor}In [{\color{incolor}26}]:} \PY{n}{plt}\PY{o}{.}\PY{n}{figure}\PY{p}{(}\PY{n}{figsize}\PY{o}{=}\PY{p}{(}\PY{l+m+mi}{16}\PY{p}{,}\PY{l+m+mi}{7}\PY{p}{)}\PY{p}{)}
         \PY{n}{Barraq}\PY{o}{.}\PY{n}{PdfsPlot}\PY{p}{(}\PY{p}{)}
\end{Verbatim}

    \begin{center}
    \adjustimage{max size={0.9\linewidth}{0.9\paperheight}}{barraq_files/barraq_27_0.png}
    \end{center}
    { \hspace*{\fill} \\}
    
    Plusieurs informations musicologiques peuvent êtes lus sur ce graphe.
Nous pouvons voir que la plupart des courbes partagent les mêmes pôles
fréquenciels que la totalité de l'extrait (\emph{P0}). Nous pouvons voir
aussi que la première phrase (\emph{P1}) appuie principalement la
fréquence dominante 244 Hz; celle-ci, avec de légères variations,
constitue la fréquence la plus présente dans la plupart des extraits.

Nous pouvons peut lire également que la troisième phrase (P3) appuie,
encore plus que la deuxième phrase, les pôles fréquenciels aïgus, alors
que la quatrième phrase appuie particulièrement la fréquence 200 Hz.

Une méthode mathématique permettant d'analyser la proximité des courbes
est le calcul des \emph{coefficient de corrélation}
(http://docs.scipy.org/doc/numpy/reference/generated/numpy.corrcoef.html),
ce qui nous donne les valeurs suivantes :

    \begin{Verbatim}[commandchars=\\\{\}]
{\color{incolor}In [{\color{incolor}12}]:} \PY{n}{Barraq}\PY{o}{.}\PY{n}{PdfCorr}
\end{Verbatim}

            \begin{Verbatim}[commandchars=\\\{\}]
{\color{outcolor}Out[{\color{outcolor}12}]:} array([[ 1.    ,  0.8945,  0.9698,  0.7143,  0.8769,  0.946 ,  0.9093],
                [ 0.8945,  1.    ,  0.8782,  0.4979,  0.8524,  0.913 ,  0.7122],
                [ 0.9698,  0.8782,  1.    ,  0.7984,  0.7916,  0.8906,  0.8133],
                [ 0.7143,  0.4979,  0.7984,  1.    ,  0.377 ,  0.5059,  0.5774],
                [ 0.8769,  0.8524,  0.7916,  0.377 ,  1.    ,  0.9005,  0.7847],
                [ 0.946 ,  0.913 ,  0.8906,  0.5059,  0.9005,  1.    ,  0.859 ],
                [ 0.9093,  0.7122,  0.8133,  0.5774,  0.7847,  0.859 ,  1.    ]])
\end{Verbatim}
        
    On peut représenter ces coefficients sous la forme d'une matrice de
similarité (\emph{self-similarity matrix}), qui nous renseigne plus
facilement sur la proximité entre les extraits :

    \begin{Verbatim}[commandchars=\\\{\}]
{\color{incolor}In [{\color{incolor}13}]:} \PY{n}{Barraq}\PY{o}{.}\PY{n}{Simatrix}\PY{p}{(}\PY{p}{)}
\end{Verbatim}

    \begin{center}
    \adjustimage{max size={0.9\linewidth}{0.9\paperheight}}{barraq_files/barraq_31_0.png}
    \end{center}
    { \hspace*{\fill} \\}
    
    Ainsi, on peut lire que la proximité la plus élevée se trouve entre
l'extrait et lui-même : 0 à 0, 1 à 1, 2 à 2, etc., (coeficient égale à
1). On peut directement lire également que les extraits les plus proches
de la totalité du chant (P0) sont les extraits 2, 5 et 6 ayant
respectivement les coefficients de corrélation 0.9697, 0.9459 et 0.9093.
L'extrait le plus éloigné étant la phrase 3, avec un coefficient de
0.71426501. Cependant, cette valeur demeure élevée -- nous sommes encore
relativement éloigné de la valeur 0--, ce qui montre que toutes les
phrases partagent les mêmes pôles fréquenciels : nous n'avons pas de
modulation dans ce chant.

    \subsection{Analyse du contenu
intervallique}\label{analyse-du-contenu-intervallique}

    \subsubsection{Détection de la
tonique}\label{duxe9tection-de-la-tonique}

    On peut obtenir la tonique en la définissant comme étant la fréquence
qui a la probabilité la plus élevée dans la dernière partie de la phrase
(\emph{method=pdf}). Comparons ici les valeurs obtenues en considérant
différents pourcentages des dernières fréquences présentes dans les
différentes phrases (0.5\%, 1\%, 1.5\%, 2\%, 5\%, 10\%, 15\%) :

    \begin{Verbatim}[commandchars=\\\{\}]
{\color{incolor}In [{\color{incolor}14}]:} \PY{k}{for} \PY{n}{i} \PY{o+ow}{in} \PY{n+nb}{range}\PY{p}{(}\PY{l+m+mi}{0}\PY{p}{,}\PY{n+nb}{len}\PY{p}{(}\PY{n}{Barraq}\PY{o}{.}\PY{n}{melodies}\PY{p}{)}\PY{p}{)}\PY{p}{:}
             \PY{n}{toniques} \PY{o}{=} \PY{p}{[}\PY{p}{]}
             \PY{k}{for} \PY{n}{j} \PY{o+ow}{in} \PY{p}{[}\PY{l+m+mf}{0.5}\PY{p}{,}\PY{l+m+mi}{1}\PY{p}{,}\PY{l+m+mf}{1.5}\PY{p}{,}\PY{l+m+mi}{2}\PY{p}{,}\PY{l+m+mi}{5}\PY{p}{,}\PY{l+m+mi}{10}\PY{p}{,}\PY{l+m+mi}{15}\PY{p}{]}\PY{p}{:}
                 \PY{n}{toniques}\PY{o}{.}\PY{n}{append}\PY{p}{(}\PY{n}{Barraq}\PY{o}{.}\PY{n}{melodies}\PY{p}{[}\PY{n}{i}\PY{p}{]}\PY{o}{.}\PY{n}{tonique}\PY{p}{(}\PY{n}{j}\PY{p}{,}\PY{n}{method}\PY{o}{=}\PY{l+s}{\PYZdq{}}\PY{l+s}{pdf}\PY{l+s}{\PYZdq{}}\PY{p}{)}\PY{p}{[}\PY{l+m+mi}{1}\PY{p}{]}\PY{p}{)}
             \PY{k}{print} \PY{l+s}{\PYZdq{}}\PY{l+s}{Toniques possibles de la phrase }\PY{l+s}{\PYZdq{}}\PY{p}{,}\PY{n}{i}\PY{p}{,}\PY{l+s}{\PYZdq{}}\PY{l+s}{ : }\PY{l+s}{\PYZdq{}}\PY{p}{,} \PY{n}{toniques}
\end{Verbatim}

    \begin{Verbatim}[commandchars=\\\{\}]
Toniques possibles de la phrase  0  :  [163, 162, 162, 181, 187, 195, 198]
Toniques possibles de la phrase  1  :  [166, 163, 162, 163, 184, 185, 189]
Toniques possibles de la phrase  2  :  [156, 157, 158, 158, 183, 184, 186]
Toniques possibles de la phrase  3  :  [312, 308, 184, 183, 186, 189, 194]
Toniques possibles de la phrase  4  :  [158, 158, 159, 158, 183, 197, 198]
Toniques possibles de la phrase  5  :  [158, 158, 157, 157, 183, 186, 189]
Toniques possibles de la phrase  6  :  [163, 162, 162, 162, 162, 184, 186]
    \end{Verbatim}

    Il est possible de définir la tonique comme étant la fréquence la plus
présente parmi les dernières fréquences (\emph{method=``mode''}) :

    \begin{Verbatim}[commandchars=\\\{\}]
{\color{incolor}In [{\color{incolor}15}]:} \PY{k}{for} \PY{n}{i} \PY{o+ow}{in} \PY{n+nb}{range}\PY{p}{(}\PY{l+m+mi}{0}\PY{p}{,}\PY{n+nb}{len}\PY{p}{(}\PY{n}{Barraq}\PY{o}{.}\PY{n}{melodies}\PY{p}{)}\PY{p}{)}\PY{p}{:}
             \PY{n}{toniques} \PY{o}{=} \PY{p}{[}\PY{p}{]}
             \PY{k}{for} \PY{n}{j} \PY{o+ow}{in} \PY{p}{[}\PY{l+m+mf}{0.5}\PY{p}{,}\PY{l+m+mi}{1}\PY{p}{,}\PY{l+m+mf}{1.5}\PY{p}{,}\PY{l+m+mi}{2}\PY{p}{,}\PY{l+m+mi}{5}\PY{p}{,}\PY{l+m+mi}{10}\PY{p}{,}\PY{l+m+mi}{15}\PY{p}{]}\PY{p}{:}
                 \PY{n}{toniques}\PY{o}{.}\PY{n}{append}\PY{p}{(}\PY{n+nb}{int}\PY{p}{(}\PY{n}{Barraq}\PY{o}{.}\PY{n}{melodies}\PY{p}{[}\PY{n}{i}\PY{p}{]}\PY{o}{.}\PY{n}{tonique}\PY{p}{(}\PY{n}{j}\PY{p}{,}\PY{n}{method}\PY{o}{=}\PY{l+s}{\PYZdq{}}\PY{l+s}{mode}\PY{l+s}{\PYZdq{}}\PY{p}{)}\PY{p}{[}\PY{l+m+mi}{1}\PY{p}{]}\PY{p}{)}\PY{p}{)}
             \PY{k}{print} \PY{l+s}{\PYZdq{}}\PY{l+s}{Toniques possibles de la phrase }\PY{l+s}{\PYZdq{}}\PY{p}{,}\PY{n}{i}\PY{p}{,}\PY{l+s}{\PYZdq{}}\PY{l+s}{ : }\PY{l+s}{\PYZdq{}}\PY{p}{,} \PY{n}{toniques}
\end{Verbatim}

    \begin{Verbatim}[commandchars=\\\{\}]
Toniques possibles de la phrase  0  :  [162, 162, 180, 180, 180, 158, 269]
Toniques possibles de la phrase  1  :  [167, 167, 162, 163, 184, 184, 183]
Toniques possibles de la phrase  2  :  [156, 158, 158, 158, 184, 245, 183]
Toniques possibles de la phrase  3  :  [312, 312, 312, 182, 182, 182, 182]
Toniques possibles de la phrase  4  :  [159, 159, 159, 157, 181, 197, 239]
Toniques possibles de la phrase  5  :  [158, 158, 158, 158, 181, 181, 244]
Toniques possibles de la phrase  6  :  [162, 162, 162, 162, 162, 180, 180]
    \end{Verbatim}

    Les deux méthodes s'accordent globalement dans les résultats obtenus
entre 0.5\% et 1\% dernières fréquences de la phrase. La fréquence
162-193 Hz est la tonique de tout le chant. Nous remarquons que cette
tonique baisse dans les deuxième, quatrième et cinquième phrases {[}La
fréquence de la tonique 312 indiquées pour la troisième phrase
correspond à l'octave supérieur par rapport à la fréquence 156 Hz. Ceci
est dû à la détection de la fréquence fondamentale par l'algorithme de
détection f0.{]}.

    \subsubsection{Tonique et fréquence dominante : intervalle
dominant}\label{tonique-et-fruxe9quence-dominante-intervalle-dominant}

    Dans un contexte modal, l'intervalle formé par la note-fréquence
dominante et la tonique est un intervalle fondamental. Observons sa
variation par rapport aux différentes phrases :

    \begin{Verbatim}[commandchars=\\\{\}]
{\color{incolor}In [{\color{incolor}16}]:} \PY{n}{TD} \PY{o}{=} \PY{p}{[}\PY{p}{]}
         \PY{k}{for} \PY{n}{i} \PY{o+ow}{in} \PY{n+nb}{range}\PY{p}{(}\PY{l+m+mi}{0}\PY{p}{,}\PY{n+nb}{len}\PY{p}{(}\PY{n}{Barraq}\PY{o}{.}\PY{n}{melodies}\PY{p}{)}\PY{p}{)}\PY{p}{:}
             \PY{n}{tonique} \PY{o}{=} \PY{n}{Barraq}\PY{o}{.}\PY{n}{melodies}\PY{p}{[}\PY{n}{i}\PY{p}{]}\PY{o}{.}\PY{n}{tonique}\PY{p}{(}\PY{o}{.}\PY{l+m+mi}{5}\PY{p}{,}\PY{n}{method}\PY{o}{=}\PY{l+s}{\PYZdq{}}\PY{l+s}{mode}\PY{l+s}{\PYZdq{}}\PY{p}{)}\PY{p}{[}\PY{l+m+mi}{1}\PY{p}{]}
             \PY{n}{peakidx} \PY{o}{=} \PY{n}{numpy}\PY{o}{.}\PY{n}{argmax}\PY{p}{(}\PY{n}{Barraq}\PY{o}{.}\PY{n}{melodies}\PY{p}{[}\PY{n}{i}\PY{p}{]}\PY{o}{.}\PY{n}{peakspdf}\PY{p}{)}
             \PY{n}{dominante} \PY{o}{=} \PY{n}{Barraq}\PY{o}{.}\PY{n}{melodies}\PY{p}{[}\PY{n}{i}\PY{p}{]}\PY{o}{.}\PY{n}{peaks}\PY{p}{[}\PY{n}{peakidx}\PY{p}{]}
             \PY{n}{TD}\PY{o}{.}\PY{n}{append}\PY{p}{(}\PY{p}{[}\PY{n}{dominante}\PY{p}{,}\PY{n}{tonique}\PY{p}{]}\PY{p}{)}
         \PY{n}{numpy}\PY{o}{.}\PY{n}{array}\PY{p}{(}\PY{n}{TD}\PY{p}{)}
\end{Verbatim}

            \begin{Verbatim}[commandchars=\\\{\}]
{\color{outcolor}Out[{\color{outcolor}16}]:} array([[ 242.4069,  162.922 ],
                [ 244.4126,  167.696 ],
                [ 243.4097,  156.466 ],
                [ 321.6332,  312.932 ],
                [ 200.2865,  159.201 ],
                [ 242.4069,  158.284 ],
                [ 271.49  ,  162.922 ]])
\end{Verbatim}
        
    Pour une lecture comparative de ces valeurs ils seraient plus judicieux
de les convertir sur une échelle linéaire; soit en savarts :

    \begin{Verbatim}[commandchars=\\\{\}]
{\color{incolor}In [{\color{incolor}17}]:} \PY{k}{for} \PY{n}{i} \PY{o+ow}{in} \PY{n+nb}{range}\PY{p}{(}\PY{l+m+mi}{0}\PY{p}{,}\PY{n+nb}{len}\PY{p}{(}\PY{n}{TD}\PY{p}{)}\PY{p}{)}\PY{p}{:}
             \PY{n}{I} \PY{o}{=} \PY{n}{numpy}\PY{o}{.}\PY{n}{float32}\PY{p}{(}\PY{n}{log10}\PY{p}{(}\PY{n}{TD}\PY{p}{[}\PY{n}{i}\PY{p}{]}\PY{p}{[}\PY{l+m+mi}{0}\PY{p}{]}\PY{o}{/}\PY{n}{TD}\PY{p}{[}\PY{n}{i}\PY{p}{]}\PY{p}{[}\PY{l+m+mi}{1}\PY{p}{]}\PY{p}{)}\PY{o}{*}\PY{l+m+mi}{1000}\PY{p}{)}
             \PY{k}{print} \PY{l+s}{\PYZdq{}}\PY{l+s}{Intervalle fondamental de la phrase}\PY{l+s}{\PYZdq{}}\PY{p}{,} \PY{n}{i}\PY{p}{,}\PY{l+s}{\PYZdq{}}\PY{l+s}{ : }\PY{l+s}{\PYZdq{}}\PY{p}{,} \PY{n}{I}\PY{p}{,} \PY{l+s}{\PYZdq{}}\PY{l+s}{savarts}\PY{l+s}{\PYZdq{}}
\end{Verbatim}

    \begin{Verbatim}[commandchars=\\\{\}]
Intervalle fondamental de la phrase 0  :  172.565 savarts
Intervalle fondamental de la phrase 1  :  163.601 savarts
Intervalle fondamental de la phrase 2  :  191.918 savarts
Intervalle fondamental de la phrase 3  :  11.9109 savarts
Intervalle fondamental de la phrase 4  :  99.706 savarts
Intervalle fondamental de la phrase 5  :  185.108 savarts
Intervalle fondamental de la phrase 6  :  221.774 savarts
    \end{Verbatim}

    Rappelons la valeur des intervalles fondamentaux en savarts :

    \begin{Verbatim}[commandchars=\\\{\}]
{\color{incolor}In [{\color{incolor}18}]:} \PY{k+kn}{from} \PY{n+nn}{diastema} \PY{k+kn}{import} \PY{n}{epimores}
         \PY{n}{epi}\PY{p}{(}\PY{n+nb}{list}\PY{o}{=}\PY{l+s}{\PYZdq{}}\PY{l+s}{Yes}\PY{l+s}{\PYZdq{}}\PY{p}{)}\PY{p}{;}
\end{Verbatim}

    \begin{Verbatim}[commandchars=\\\{\}]
2/1*4/3  ::  425.969 s.
4/3  ::  124.939 s.
12/11  ::  37.7886 s.
1/1  ::  0.0 s.
6/5  ::  79.1812 s.
3/2*6/5  ::  255.273 s.
9/8*12/11  ::  88.9411 s.
3/2*5/4  ::  273.001 s.
2/1*12/11  ::  338.819 s.
3/2*9/8  ::  227.244 s.
2/1*9/8  ::  352.183 s.
2/1*6/5  ::  380.211 s.
3/2  ::  176.091 s.
5/4  ::  96.91 s.
3/2*10/9  ::  221.849 s.
2/1  ::  301.03 s.
10/9  ::  45.7575 s.
2/1*10/9  ::  346.787 s.
9/8  ::  51.1525 s.
2/1*5/4  ::  397.94 s.
    \end{Verbatim}

    Nous pouvons voir que dans la plupart des phrases, l'intervalle
fondamental est une quinte, dont la valeur est très variable. La phrase
P4 s'appuie sur une tierce 5/4 et la dernière phrase sur une sixte.

    \subsubsection{Échelle par rapport à la
tonique}\label{uxe9chelle-par-rapport-uxe0-la-tonique}

    \begin{Verbatim}[commandchars=\\\{\}]
{\color{incolor}In [{\color{incolor}19}]:} \PY{n}{Barraq}\PY{o}{.}\PY{n}{melodies}\PY{p}{[}\PY{l+m+mi}{0}\PY{p}{]}\PY{o}{.}\PY{n}{get\PYZus{}intervals}\PY{p}{(}\PY{p}{)}
\end{Verbatim}

            \begin{Verbatim}[commandchars=\\\{\}]
{\color{outcolor}Out[{\color{outcolor}19}]:} array([ 172.5652,   89.672 ,  224.9708,   -1.9887,  129.14  ,  295.3812,
                 260.1352,  351.2157,  395.337 ,  426.4666])
\end{Verbatim}
        
    \subsubsection{Échelles de plusieurs
extraits}\label{uxe9chelles-de-plusieurs-extraits}

    \begin{Verbatim}[commandchars=\\\{\}]
{\color{incolor}In [{\color{incolor}20}]:} \PY{n}{Barraq}\PY{o}{.}\PY{n}{Intervals}\PY{p}{(}\PY{p}{)}
\end{Verbatim}

            \begin{Verbatim}[commandchars=\\\{\}]
{\color{outcolor}Out[{\color{outcolor}20}]:} [array([ 172.5652,   89.672 ,  224.9708,   -1.9887,  129.14  ,  295.3812,
                  260.1352,  351.2157,  395.337 ,  426.4666]),
          array([ 163.6009,   81.4566,  124.468 ,   -1.3073,  289.5567,  220.3181,
                  352.732 , -199.1464,  384.9402]),
          array([ 191.918 ,  111.5593,  283.5301,  316.9845,   75.6679,  248.8542,
                  154.5707,   18.2483,  367.583 ,  398.6819]),
          array([  11.9109,  -20.4077, -109.112 ,  -53.7481, -195.9783, -285.459 ,
                   67.7454,  112.9412,  142.9964]),
          array([  99.706 ,  182.5991,   10.7225,   65.7973,  143.1273,  233.4093,
                  265.7407,  301.3336,  368.3361]),
          array([ 185.1079,  100.0347,  234.3168,    5.1494,  263.7755,  307.9239,
                  366.1334]),
          array([ 221.7741,  170.7647,   87.492 ,   -7.3934,  121.1238,   48.652 ,
                  289.9305, -132.5911,  347.6285])]
\end{Verbatim}
        
    \subsection{Une fonction de densité de probabilité comme échelle
générale}\label{une-fonction-de-densituxe9-de-probabilituxe9-comme-uxe9chelle-guxe9nuxe9rale}

    Afin de constituer une échelle générale du chant, nous pourrions définir
une fonction de densité de probabilité comme étant la somme des
différentes fonctions. Nous pouvons comparer cette fonction à celle
obtenu par la totalite du chant, P0 :

    \begin{Verbatim}[commandchars=\\\{\}]
{\color{incolor}In [{\color{incolor}21}]:} \PY{n}{PDF} \PY{o}{=} \PY{p}{[}\PY{p}{]}
         \PY{k}{for} \PY{n}{i} \PY{o+ow}{in} \PY{n+nb}{range}\PY{p}{(}\PY{l+m+mi}{1}\PY{p}{,}\PY{n+nb}{len}\PY{p}{(}\PY{n}{Barraq}\PY{o}{.}\PY{n}{melodies}\PY{p}{)}\PY{p}{)}\PY{p}{:}
             \PY{n}{PDF} \PY{o}{=}\PY{o}{+} \PY{n}{Barraq}\PY{o}{.}\PY{n}{melodies}\PY{p}{[}\PY{n}{i}\PY{p}{]}\PY{o}{.}\PY{n}{pdf}
         
         \PY{n}{PDF\PYZus{}norm} \PY{o}{=} \PY{n}{PDF}\PY{o}{*}\PY{l+m+mi}{1}\PY{o}{/}\PY{n+nb}{max}\PY{p}{(}\PY{n}{PDF}\PY{p}{)}
         \PY{n}{P0\PYZus{}norm} \PY{o}{=} \PY{n}{Barraq}\PY{o}{.}\PY{n}{melodies}\PY{p}{[}\PY{l+m+mi}{0}\PY{p}{]}\PY{o}{.}\PY{n}{pdf}\PY{o}{*}\PY{p}{(}\PY{l+m+mi}{1}\PY{o}{/}\PY{n+nb}{max}\PY{p}{(}\PY{n}{Barraq}\PY{o}{.}\PY{n}{melodies}\PY{p}{[}\PY{l+m+mi}{0}\PY{p}{]}\PY{o}{.}\PY{n}{pdf}\PY{p}{)}\PY{p}{)}
         \PY{n}{plt}\PY{o}{.}\PY{n}{plot}\PY{p}{(}\PY{n}{PDF\PYZus{}norm}\PY{p}{,} \PY{n}{label}\PY{o}{=}\PY{l+s}{\PYZdq{}}\PY{l+s}{Global PDF}\PY{l+s}{\PYZdq{}}\PY{p}{)}
         \PY{n}{plt}\PY{o}{.}\PY{n}{plot}\PY{p}{(}\PY{n}{P0\PYZus{}norm}\PY{p}{,} \PY{n}{label}\PY{o}{=}\PY{l+s}{\PYZdq{}}\PY{l+s}{P0 pdf}\PY{l+s}{\PYZdq{}}\PY{p}{)}
         \PY{n}{plt}\PY{o}{.}\PY{n}{legend}\PY{p}{(}\PY{p}{)}
\end{Verbatim}

            \begin{Verbatim}[commandchars=\\\{\}]
{\color{outcolor}Out[{\color{outcolor}21}]:} <matplotlib.legend.Legend at 0x1124d0f90>
\end{Verbatim}
        
    \begin{center}
    \adjustimage{max size={0.9\linewidth}{0.9\paperheight}}{barraq_files/barraq_54_1.png}
    \end{center}
    { \hspace*{\fill} \\}
    
    Nous remarquons que la PDF globale, somme de toutes les fonctions de
densité de probabilité : comme elle prend en considération les
différentes variations de la tonique selon les phrases, les pôles
fréquenciels se trouvent affirmés. C'est donc cette fonction que nous
prendrons en considération pour la détermination de l'échelle globale.
Ses pics peuvent être obtenus comme suit :

    \begin{Verbatim}[commandchars=\\\{\}]
{\color{incolor}In [{\color{incolor}22}]:} \PY{n}{P} \PY{o}{=} \PY{p}{(}\PY{p}{(}\PY{n}{numpy}\PY{o}{.}\PY{n}{diff}\PY{p}{(}\PY{n}{numpy}\PY{o}{.}\PY{n}{sign}\PY{p}{(}\PY{n}{numpy}\PY{o}{.}\PY{n}{diff}\PY{p}{(}\PY{n}{PDF}\PY{p}{)}\PY{p}{)}\PY{p}{)} \PY{o}{\PYZlt{}} \PY{l+m+mi}{0}\PY{p}{)}\PY{o}{.}\PY{n}{nonzero}\PY{p}{(}\PY{p}{)}\PY{p}{[}\PY{l+m+mi}{0}\PY{p}{]} \PY{o}{+} \PY{l+m+mi}{1}\PY{p}{)}\PY{o}{+}\PY{l+m+mi}{100} \PY{c}{\PYZsh{} local max}
         \PY{n}{P}
\end{Verbatim}

            \begin{Verbatim}[commandchars=\\\{\}]
{\color{outcolor}Out[{\color{outcolor}22}]:} array([120, 160, 182, 199, 215, 241, 271, 317, 362])
\end{Verbatim}
        
    Ainsi, \textbf{l'échelle musicale de tout le chant, en savarts} peut
donc être représentée par les intervalles :

    \begin{Verbatim}[commandchars=\\\{\}]
{\color{incolor}In [{\color{incolor}23}]:} \PY{n}{Echelle} \PY{o}{=} \PY{p}{[}\PY{p}{]}
         \PY{k}{for} \PY{n}{i} \PY{o+ow}{in} \PY{n+nb}{range}\PY{p}{(}\PY{l+m+mi}{0}\PY{p}{,}\PY{n+nb}{len}\PY{p}{(}\PY{n}{P}\PY{p}{)}\PY{p}{)}\PY{p}{:}
             \PY{n}{Echelle}\PY{o}{.}\PY{n}{append}\PY{p}{(}\PY{n}{log10}\PY{p}{(}\PY{n}{P}\PY{p}{[}\PY{n}{i}\PY{p}{]}\PY{o}{/}\PY{l+m+mf}{160.}\PY{p}{)}\PY{o}{*}\PY{l+m+mi}{1000}\PY{p}{)} \PY{c}{\PYZsh{} 160 Hz étant la tonique}
\end{Verbatim}

    La méthode \emph{Intervs()} nous permet de comparer contenu de cette
échelle en comparant les intervalles en savarts aux valeurs des
intervalles de références :

    \begin{Verbatim}[commandchars=\\\{\}]
{\color{incolor}In [{\color{incolor}24}]:} \PY{n}{Inters}\PY{p}{(}\PY{n}{Echelle}\PY{p}{)}
\end{Verbatim}

            \begin{Verbatim}[commandchars=\\\{\}]
{\color{outcolor}Out[{\color{outcolor}24}]:} [['-124.94', '4/3', '-', '0.00'],
          ['0.00', '1/1', '+', '0.00'],
          ['55.95', '9/8', '+', '4.80'],
          ['94.73', '5/4', '-', '2.18'],
          ['128.32', '4/3', '+', '3.38'],
          ['177.90', '3/2', '+', '1.81'],
          ['228.85', '3/2*9/8', '+', '1.61'],
          ['296.94', '2/1', '-', '4.09'],
          ['354.59', '2/1*9/8', '+', '2.41']]
\end{Verbatim}
        
    Ainsi, selon cette approche, dans ce chant ʿAlī al-Barrāq, la quarte au
dessous de la tonique est très exacte. Le ton est un peu large (plus
grave que le ton 9/8 de 4.8 savarts). C'est l'intervalle le plus éloigné
de l'intervalle de référence. Sachant que l'intervalle épimore le plus
proche de cet écart est l'intervalle 91/90, lui-même plus petit qu'un
comma syntonique (81/80), on peut dire que ces intervalles représentent
de manière assez précise l'échelle globale utilisée dans ce chant.

    \subsection*{Conclusion}\label{conclusion}

    Nous venons de voir comment il est possible d'utiliser certains concepts
statistiques et musicologiques, que nous avons implémentés dans le
module Python Diastema, afin de développer une méthode d'analyse des
intervalles mélodiques. L'avantage de cette méthode est de pouvoir
comparer de manière relativement objective et rapide un ensemble de
mélodies, sur la base de leur échelle. Étant donnée que l'échelle est
conçu comme une probabilité de présence des fréquences, les pics de de
la courbe donnée par les différentes fonctions nous renseignent sur les
pôles fréquenciels présents dans la mélodie. Pour une appréciation de ce
contenu fréquentiels, il est nécessaire de retourner à des concepts
musicologiques, comme la tonique. Nous avons vu que les résultats
peuvent varier selon les paramètres utilisés pour la détection de la
fréquence représentative de la tonique d'une phrase. Finalement,
l'échelle globale d'une mélodie segmentée peut-être obtenue à partir de
la somme des fonctions des densités de probabilité.


    % Add a bibliography block to the postdoc
    
    
    
    \end{document}
